\section{Escenarios}
\subsection{Disponibilidad}
\textbf{Motivación}
\begin{itemize}
 \item Del Ministerio de Agricultura se quiere que el sistema funcione ``todo el tiempo, en todos los climas''.
 \item A su vez, se niega a utilizar la red privada salvo que sea estrictamente necesario.
 \item El Repesentante de Entidades y Productores Agrícolas quiere poder usarlo ``en todo el país en todo momento''.
 \item Defensa al Consumidor coincide, el sistema debe funcionar ``en todo el país sin problemas''.
 \item Los drones de la red privadad mejoran la disponibilidad.
 \item El sistema se usa en una zona donde no se puede garantizar conectivdad permanente entre los distntos artefactos.
\end{itemize}


\begin{tabular}{| l || p{12cm} |}
\hline 
\textbf{Descripci\'on} & La red de drones pública podría tener problemas. En ese caso, hay que pasar a usar la red privada. \\
\hline 
\textbf{Fuente} & Pedidos de imágenes \\
\hline 
\textbf{Estimulo} & Un pedido de foto de una posición \\
\hline 
\textbf{Artefacto} & Sistema de comunicación con drones \\
\hline 
\textbf{Entorno} & Conexión con la red de drones pública caída \\
\hline 
\textbf{Respuesta} & La foto se saca de todos modos usando la red privada \\
\hline 
\textbf{Medici\'on} & 9 de cada 10 veces la foto se saca correctamente \\
\hline 
\end{tabular}

\medskip

\begin{tabular}{| l || p{12cm} |}
\hline 
\textbf{Descripci\'on} & Si la red pública de drones se reestablece, debo priorizarla para ahorrar recursos estatales. \\
\hline 
\textbf{Fuente} & Pedidos de imágenes \\
\hline 
\textbf{Estimulo} & Un pedido de foto de una posición \\
\hline 
\textbf{Artefacto} & Sistema de comunicación con drones \\
\hline 
\textbf{Entorno} & Conexión con la red de drones pública funcionando correctamente \\
\hline 
\textbf{Respuesta} & La foto se saca usando la red pública \\
\hline 
\textbf{Medici\'on} & La foto se obtiene correctamente \\
\hline 
\end{tabular}

\medskip

\begin{tabular}{| l || p{12cm} |}
\hline 
\textbf{Descripci\'on} & Los servidores del $INTA$ para el pronóstico del tiempo son nuevos y podr\'ian fallar. Se deben comunicar las fallas de los servidores al $INTA$. \\
\hline 
\textbf{Fuente} & Interfaz Estación Meteorológica\\
\hline 
\textbf{Estimulo} & Los servidores del $INTA$ fallan\\
\hline 
\textbf{Artefacto} & Comunicador Estación Meteorológica \\
\hline 
\textbf{Entorno} & Normal \\
\hline 
\textbf{Respuesta} & Se envía un mail al responsable técnico del $INTA$ anunciando la falla \\
\hline 
\textbf{Medici\'on} & El mail se envía en los siguientes 5 minutos \\
\hline 
\end{tabular}

\medskip

\begin{tabular}{| l || p{12cm} |}
\hline 
\textbf{Descripci\'on} & Tanto el Responsable T\'ecnico de ArSAT como el Representante de Entidades y Productores Agr\'icolas están preocupados por que el volumen de informaci\'on sea demasiado. El sistema debe responder bien ante grandes cantidades de datos. \\
\hline 
\textbf{Fuente} & Interfaz con los drones \\
\hline 
\textbf{Estimulo} & Se reciben las imágenes de los drones \\
\hline 
\textbf{Artefacto} & Sistema de procesamiento de imágenes \\
\hline 
\textbf{Entorno} & Normal \\
\hline 
\textbf{Respuesta} & Se procesan las imágenes correctamente \\
\hline 
\textbf{Medici\'on} & En un 99\% de los casos se procesan correctamente las imágenes \\
\hline 
\end{tabular}

\subsection{Performance}
\textbf{Motivación}
\begin{itemize}
 \item El Ministerio de Agricultura quiere que el sistema funcione rápido porque sino podría arruinar cosechas.
 \item El Responsable Técnico prefiere usar los drones privados porque garantizan mayor rapidez.
 \item Pero el Ministerio de Agricultura se opone.
 \item El enunciado especifica que la interacción con los actuadores debe ser rápida.
\end{itemize}

\begin{tabular}{| l || p{12cm} |}
\hline 
\textbf{Descripci\'on} & La interacción con los actuadores debe ser rápida \\
\hline 
\textbf{Fuente} & Planificador \\
\hline 
\textbf{Estimulo} & Se envía una orden de riego \\
\hline 
\textbf{Artefacto} & Manejador de actuadores \\
\hline 
\textbf{Entorno} & Normal \\
\hline 
\textbf{Respuesta} & Los actuadores riegan la cosecha \\
\hline 
\textbf{Medici\'on} & El riego comienza a los 20 segundos de enviada la órden \\
\hline 
\end{tabular}

\medskip

\begin{tabular}{| l || p{12cm} |}
\hline 
\textbf{Descripci\'on} & Se debe priorizar el uso de drones públicos, pero pasar a los drones privados en caso de que los públicos tarden en responder \\
\hline 
\textbf{Fuente} & Interfaz con los drones \\
\hline 
\textbf{Estimulo} & Una respuesta de la red de drones pública toma más de 10 segundos \\
\hline 
\textbf{Artefacto} & Sistema de comunicación con drones \\
\hline 
\textbf{Entorno} & Normal \\
\hline 
\textbf{Respuesta} & Se empieza a utilizar la red de drones privada \\
\hline 
\textbf{Medici\'on} & El cambio del sistema se realiza en menos de 20 segundos \\
\hline 
\end{tabular}


\subsection{Seguridad}
\textbf{Motivación}
\begin{itemize}
 \item El Responsable T\'ecnico de ArSAT quiere ``un esquema serio de accesos y permisos'' para los usuarios.
 \item El Representante de Entidades y Productores Agr\'icolas considera fundamental que los planes maestros y datos de seguimiento sean seguros y confidenciales.
 \item Defensa al Consumidor quiere asegurar confidencialidad de todos los datos involucrados.
\end{itemize}

\begin{tabular}{| l || p{12cm} |}
\hline 
\textbf{Descripci\'on} & Un atacante debe ser rechazado \\
\hline 
\textbf{Fuente} & Gestor de configuración \\
\hline 
\textbf{Estimulo} & Un usuario intenta consultar un plan maestro que no le pertenece \\
\hline 
\textbf{Artefacto} & Gestor de autenticación \\
\hline 
\textbf{Entorno} & Normal \\
\hline 
\textbf{Respuesta} & Se rechaza la solicitud y se guarda un registro del intento \\
\hline 
\textbf{Medici\'on} & La solicitud se rechaza un 99,99\% de las veces \\
\hline 
\end{tabular}

\medskip

\begin{tabular}{| l || p{12cm} |}
\hline 
\textbf{Descripci\'on} & Un usuario correctamente identificado debe poder acceder a sus datos \\
\hline 
\textbf{Fuente} & Sistema de planes maestros \\
\hline 
\textbf{Estimulo} & Un usuario intenta consultar sus propios planes maestros \\
\hline 
\textbf{Artefacto} & Gestor de autenticación \\
\hline 
\textbf{Entorno} & Normal \\
\hline 
\textbf{Respuesta} & Se permite al usuario acceder \\
\hline 
\textbf{Medici\'on} & La solicitud se acepta un 99,99\% de las veces \\
\hline 
\end{tabular}

\subsubsection{Auditabilidad}
\textbf{Motivación}
\begin{itemize}
 \item El Representante de Entidades y Productores Agr\'icolas quiere poder supervisar las órdenes enviadas a los actuadores.
 \item Defensa al Consumidor quiere validar la imparcialidad del sistema, para lo cual quiere poder auditar todas las decisiones tomadas.
 \item También quiere asegurarse que los drones no serán utilizados para espiar a la población, con lo cual quiere poder conocer las instrucciones que se le dan y la información que devuelven.
 \item Sin embargo esto entra en conflicto con atributos de disponibilidad y performance juzgados más prioritarios (por ejemplo, la saturación de los servidores del $INTA$).
 \item Concluimos que se guardará siempre registro de las decisiones humanas (por ser las más importantes) y de las órdenes enviadas a los actuadores cuando el sistema tenga espacio suficiente.
\end{itemize}

\begin{tabular}{| l || p{12cm} |}
\hline 
\textbf{Descripci\'on} & Se podrán auditar las decisiones tomadas por personas \\
\hline 
\textbf{Fuente} & Gestor de configuración \\
\hline 
\textbf{Estimulo} & Un usuario cambia un plan maestro \\
\hline 
\textbf{Artefacto} & Sistema de logging \\
\hline 
\textbf{Entorno} & Normal \\
\hline 
\textbf{Respuesta} & Se guarda un log del acceso y del cambio realizado \\
\hline 
\textbf{Medici\'on} & El log se guarda correctamente en un 99,99\% de los casos \\
\hline 
\end{tabular}

\medskip

\begin{tabular}{| l || p{12cm} |}
\hline 
\textbf{Descripci\'on} & Las órdenes enviadas a los actuadores no se guardarán cuando haya poco espacio disponible \\
\hline 
\textbf{Fuente} & Mánager de Monitoreo de imágenes \\
\hline 
\textbf{Estimulo} & Se envía un pedido de captura de imágenes a una red \\
\hline 
\textbf{Artefacto} & Sistema de logging \\
\hline 
\textbf{Entorno} & Disco rígido al 90\% de su capacidad \\
\hline 
\textbf{Respuesta} & Se descarta el log de la orden enviada \\
\hline 
\textbf{Medici\'on} & El disco guarda únicamente los resultados de la orden \\
\hline 
\end{tabular}


\subsection{Modificabilidad}
\textbf{Motivación}
\begin{itemize}
 \item El Representante de Entidades y Productores Agr\'icolas quiere poder agregar nuevos actuadores sin mucha dificultad
\end{itemize}

\begin{tabular}{| l || p{12cm} |}
\hline 
\textbf{Descripci\'on} & Se pueden agregar actuadores fácilmente \\
\hline 
\textbf{Fuente} & Interfaz de configuración \\
\hline 
\textbf{Estimulo} & Un usuario desea agregar un nuevo actuador \\
\hline 
\textbf{Artefacto} & Panel de configuración de actuadores \\
\hline 
\textbf{Entorno} & Normal \\
\hline 
\textbf{Respuesta} & Se agrega el nuevo actuador correctamente \\
\hline 
\textbf{Medici\'on} & El agregado toma 5 horas hombre o menos \\
\hline 
\end{tabular}

\subsection{Usabilidad}
\textbf{Motivación}
\begin{itemize}
 \item El Representante de Entidades y Productores Agr\'icolas quiere que el sistema sea f\'acil de usar para todos los productores y los distintos tipos de perfiles
\end{itemize}

\begin{tabular}{| l || p{12cm} |}
\hline 
\textbf{Descripci\'on} & Un usuario debe poder consultar el estado de sus plantas fácilmente \\
\hline 
\textbf{Fuente} & Interfaz de usuario \\
\hline 
\textbf{Estimulo} & Un usuario quiere consultar el estado de las plantas \\
\hline 
\textbf{Artefacto} & Interfaz de usuario \\
\hline 
\textbf{Entorno} & Normal \\
\hline 
\textbf{Respuesta} & Se presentan al usuario las plantas presentes en sus campos junto con su información \\
\hline 
\textbf{Medici\'on} & No se necesitan más de tres clics para acceder a la información \\
\hline 
\end{tabular}