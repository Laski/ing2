\section{Análisis de Riesgos}
\textbf{Riesgo 1}

Desde el ministerio de agricultura se exigi\'o que en dos meses tiene que estar en funcionamiento la interacci\'on con los drones.\\
Dado que el equipo que desarrolla de esta aplicaci\'on no tiene conocimientos sobre est\'as, se decidi\'o que esto era un importante riesgo para el proyecto.

\begin{itemize}
\item \textsl{Descripci\'on}: En dos meses tienen que estar en funcionamiento la interacción con los drones estatales.
 \item \textsl{Probabilidad}: Alta
 \item \textsl{Impacto}: Alto
 \item \textsl{Exposición}: Alta
 \item \textsl{Mitigación}: Planificar los dos primeros meses focalizando solo en este problema, priorizando tareas mas riesgosas.Cada tarea deber\'ia intentar resolverse de la forma mas simple y menos general. Aislar el o los  componentes de la arquitectura que implementan la interfaz. Si es necesario hardcodear otros componentes o partes de componentes.
 \item \textsl{Plan de Contingencia} : Contratar a una empresa con experiencia en la creaci\'on de interfaces para implementar la interfaz en dos meses o comprar una si existen.
 \end{itemize}
 
\textbf{Riesgo 2}

 El atributo de calidad mas importante es el de disponibilidad y dado que el uso de la red de drones afecta de manera importante a esta se decidi\'o que era un riesgo.\\
 Tambi\'en Desde el ministerio de agricultura se quiere usar la red estatal y adem\'as el uso de la red privada tiene un costo.
 
 \begin{itemize}
 \item \textsl{Descripci\'on}: La red privada funciona mediante un sistema de abono de datos.
 \item \textsl{Probabilidad}: Media
 \item \textsl{Impacto}: Medio
 \item \textsl{Exposición}: Media
 \item \textsl{Mitigación}: Utilizar un algoritmo para maximizar uso de la red publica en las regiones.
 \item \textsl{Plan de Contingencia}: En el caso de excesivos consumos se tiene la posibilidad de desactivar el uso de la red privada.
\end{itemize}

\textbf{Riesgo 3}

La falla de los actuadores son un punto critico para el correcto funcionamiento del sistema, si estos fallan el usuario final pensara que todo el sistema esta mal.

\begin{itemize}
 \item \textsl{Descripci\'on}: La interacci\'n con los actuadores debe ser r\'apida. No se deben admitir demoras de ning\'un tipo.
 \item \textsl{Probabilidad}: Media 
 \item \textsl{Impacto}: Alto
 \item \textsl{Exposición}: Alta
 \item \textsl{Mitigación}: En la arquitectura se utilizaran t\'ecnicas para dar prioridad a las ordenes enviadas y as\'i evitar que otros procesos o acciones generen una demora con la comunicaci\'on de los actuadores. Revisar parametros de los sensores para detectar fallas.
 \item \textsl{Plan de Contingencia} : Se deben revisar peri\'odicamente los actuadores y realizar mantenimiento de estos.
\end{itemize}

\textbf{Riesgo 4}

El acceso no autorizado a la red de nodos distribuida f\'isica o no, puede afectar gravemente a la filtraci\'on de datos por este motivo es importante considerar este riesgo.

\begin{itemize}
 \item \textsl{Descripci\'on}: Todos los datos de seguimiento deben ser seguros y confidenciales para evitar filtrar datos a la competencia.
 \item \textsl{Probabilidad}: Baja
 \item \textsl{Impacto}: Medio
 \item \textsl{Exposición}: Media
 \item \textsl{Mitigación}: Encriptar todos los datos almacenados y comunicaci\'on entre los nodos. Intentar agregar un nivel adicional de seguridad para la ubicaci\'on de la clave secreta.
 \item \textsl{Plan de Contingencia} : 
\end{itemize}

\textbf{Riesgo 5}

Nuestro equipo de desarrollo piensa que el volumen de datos puede ser grande debido a la gran cantidad de informaci\'on proveniente de distintas fuentes.\\
Hacemos énfasis en la gran cantidad de imagenes que el sistema tiene que procesar.

\begin{itemize}
 \item \textsl{Descripci\'on}: No se quiere que el volumen de la informaci\'on sea un problema. No sea cosa que se empiece a trabar todo cuando lleguen las fotitos de los drones o cuando crezcan los datos al crecer las plantitas.
 \item \textsl{Probabilidad}: Media
 \item \textsl{Impacto}: Alto	
 \item \textsl{Exposición}: Baja
 \item \textsl{Mitigación}: Utilizar una infraestructura de nodos distribuidos por todo el país. Utilizar alguna tecnica para guardar informacion geografica
 \item \textsl{Plan de Contingencia} : En el caso de volumen excesivo agregar mas nodos. Tambi\'en tener mantenimiento peri\'odico
\end{itemize}

\textbf{Riesgo 6}

Como pueden existir motivos pol\'iticos en algunas zonas, necesitamos que el sistema reporte y tenga la posibilidad de guardar las fallas. \\
Si existiera la situaci\'on donde se excusa por alguna raz\'on t\'ecnica, cuando es pol\'itica, se podr\'ia justificar o tener pruebas de que esto no es cierto.

\begin{itemize}
 \item \textsl{Descripci\'on}: Zonas relegadas por motivos politicos alegando motivos t\'ecnicos
 \item \textsl{Probabilidad}: Media
 \item \textsl{Impacto}: Alto
 \item \textsl{Exposición}: Baja
 \item \textsl{Mitigación}: Reportar el estado del sistema visualmente para evitar falsas acusaciones. Tambi\'en guardar registros de fallas t\'ecnicas
 \item \textsl{Plan de Contingencia} : En caso de que se presenten casos, se deberia auditar el sistema y utilizar los logs como prueba de que el sistema funciono correctamente. 
\end{itemize}

\textbf{Riesgo 7}

Desde el \textbf{INTA} se nos informo que los servidores, como son nuevos, pueden fallar bastante.

\begin{itemize}
 \item \textsl{Descripci\'on}: Pueden surgir errores en la comunicaci\'on con los servidores del \textbf{INTA}
 \item \textsl{Probabilidad}: Media
 \item \textsl{Impacto}: Media
 \item \textsl{Exposición}: Media
 \item \textsl{Mitigación}: Detectar fallas en la comunicación.
 \item \textsl{Plan de Contingencia} : Reportar las fallas al \textbf{INTA} para que a futuro se solucionen
\end{itemize}

\textbf{Riego 8}

Como la cobertura del sistema tiene que ser en todo el pa\'is inicialmente y dado que mucha zonas son de dif\'icil acceso, esto puede traer problemas a la hora de tener el sistema funcionando en esas zonas.

\begin{itemize}
 \item \textsl{Descripci\'on}: En la zona patagónica no se puede garantizar una conectividad permanente de la red de datos entre los distintos artefactos afectados al sistema,
 \item \textsl{Probabilidad}: Alta
 \item \textsl{Impacto}: Media
 \item \textsl{Exposición}: Baja
 \item \textsl{Mitigación}: Permitir la carga manual de datos por medio de importación de archivos u otras fuentes.
 \item \textsl{Plan de Contingencia}: Considerar el uso de comunicaci\'on satelital para zonas de dif\'icil acceso. Para los dispositivos alguna t\'ecnologia alternativa de largo alcance.
\end{itemize}
