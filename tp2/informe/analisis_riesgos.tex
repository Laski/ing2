\section{Análisis de Riesgos}

\textbf{Errores en las comunicaciones}
\begin{itemize}
 \item \textsl{Descripci\'on}: Pueden surgir errores en la comunicación con los servidores del \textbf{INTA}
 \item \textsl{Probabilidad}: Media
 \item \textsl{Impacto}: Media
 \item \textsl{Exposición}: Media
 \item \textsl{Mitigación}: Detectar fallas en la comunicación para poder reportar y seguir con el normal funcionamineto.
 \item \textsl{Plan de Contingencia} : Utilizar informaci\'on de sensores e imágenes de drones para poder calcular resultados similares a los del \textbf{INTA}.
\end{itemize}

\textbf{Cobertura de conexión}
\begin{itemize}
 \item \textsl{Descripci\'on}: Problemas de conectividad en zonas sin acceso a internet u otras redes de comunicación.
 \item \textsl{Probabilidad}: Alta
 \item \textsl{Impacto}: Media
 \item \textsl{Exposición}: Media
 \item \textsl{Mitigación}: El sistema llevaría un historial con fecha de datos externos y detectaría cuándo los datos son viejos usando un reloj interno.
 \item \textsl{Plan de Contingencia}: Permitir la carga manual de datos por medio de importación de archivos u otras fuentes.
\end{itemize}

\textbf{Vulnerabilidades en la seguridad}
\begin{itemize}
 \item \textsl{Descripci\'on}: Cliente puede ver datos de otro cliente.
 \item \textsl{Probabilidad}: Baja
 \item \textsl{Impacto}: Alto
 \item \textsl{Exposición}: Media
 \item \textsl{Mitigación}: Realizar test para probar casos de borde con los cuentas de clientes. Pagar a una consultora de seguridad para pruebas tercerizadas.
 \item \textsl{Plan de Contingencia}: Permitir el apagado rápido del sistema en caso de detecci\'on de problemas de seguridad. Guardar en logs las operaciones realizadas para cada dominio / cliente / empresa.
\end{itemize}

\textbf{Aumento de costos por utilizar la red privada de drones}
\begin{itemize}
 \item \textsl{Descripci\'on}: Se utiliza mucho la red privada de drones en vez de la pública.
 \item \textsl{Probabilidad}: Media
 \item \textsl{Impacto}: Alto
 \item \textsl{Exposición}: Alto
 \item \textsl{Mitigación}: El sistema calcularía cada cierto intervalo el consumo por hect\'areas.
 \item \textsl{Plan de Contingencia} : Desactivar la red privada en caso de susperar algún límite de uso diario, mensual, etc. Enviar alertar al administrador del sistema. Si no logra acceder a la red pública pasa al caso de falta de conectividad y se aplica el plan 
\end{itemize}

\clearpage

\textbf{Delays en accionar de actuadores}
\begin{itemize}
 \item \textsl{Descripci\'on}: Demora al enviar acciones a los actuadores.
 \item \textsl{Probabilidad}: Media
 \item \textsl{Impacto}: Alta
 \item \textsl{Exposición}: Alta
 \item \textsl{Mitigación}: En el sistema se usarán actuadores que confirman la recepci\'on de acciones.
 \item \textsl{Plan de Contingencia}: Se chequeará que las acciones fueron recibidas por los actuadores y en caso de timeout se reenviarán. Ante la primera falla se le avisar\'a al administrador / encargado.
\end{itemize}

\textbf{Problemas con almacenamiento}
\begin{itemize}
 \item \textsl{Descripci\'on}: Volumen de informaci\'on muy grande.
 \item \textsl{Probabilidad}: Media
 \item \textsl{Impacto}: Alto
 \item \textsl{Exposición}: Alto
 \item \textsl{Mitigación}: Utilizar algoritmos para comprimir los datos.
 \item \textsl{Plan de Contingencia}: Monitorear el trafico de la red y espacio de disco en los nodos de \textbf{ArSAT}. Rediseñar parte de la persistencia para evitar redundancia de datos innecesaria.
\end{itemize}
