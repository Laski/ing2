\section{Análisis de Riesgos}


\textbf{Riesgo 1}
\begin{itemize}
\item \textsl{Descripci\'on}: En dos meses tienen que estar en funcionamiento la interacción con los drones estatales.
 \item \textsl{Probabilidad}: Alta
 \item \textsl{Impacto}: Alto
 \item \textsl{Exposición}: Alta
 \item \textsl{Mitigación}: Planificar los dos primeros meses focalizando solo en este problema, priorizando tareas mas riesgosas.Cada tarea deber\'ia intentar resolverse de la forma mas simple y menos general. Aislar el o los  componentes de la arquitectura que implementan la interfaz. Si es necesario hardcodear otros componentes o partes de componentes.
 \item \textsl{Plan de Contingencia} : Contratar a una empresa con experiencia en la creaci\'on de interfaces para implementar la interfaz en dos meses o comprar una si existen.
 \end{itemize}


 \textbf{Riesgo 2}
\begin{itemize}
 \item \textsl{Descripci\'on}: La red privada funciona mediante un sistema de abono de datos.
 \item \textsl{Probabilidad}: Media
 \item \textsl{Impacto}: Medio
 \item \textsl{Exposición}: Media
 \item \textsl{Mitigación}: Utilizar un algoritmo para maximizar uso de la red publica en las regiones.
 \item \textsl{Plan de Contingencia}: En el caso de excesivos consumos se tiene la posibilidad de desactivar el uso de la red privada.
\end{itemize}

 
\textbf{Riesgo 3}
\begin{itemize}
 \item \textsl{Descripci\'on}: La interacci\'n con los actuadores debe ser r\'apida. No se deben admitir demoras de ning\'un tipo.
 \item \textsl{Probabilidad}: Media 
 \item \textsl{Impacto}: Alto
 \item \textsl{Exposición}: Alta
 \item \textsl{Mitigación}: En la arquitectura se utilizaran t\'ecnicas para dar prioridad a las ordenes enviadas y as\'i evitar que otros procesos o acciones generen una demora con la comunicaci\'on de los actuadores.
 \item \textsl{Plan de Contingencia} : 
\end{itemize}

\textbf{Riesgo 4}
\begin{itemize}
 \item \textsl{Descripci\'on}: Todos los datos de seguimiento deben ser seguros y confidenciales para evitar filtrar datos a la competencia.
 \item \textsl{Probabilidad}: Baja
 \item \textsl{Impacto}: Alto
 \item \textsl{Exposición}: Media
 \item \textsl{Mitigación}: Encriptar todos los datos almacenados y comunicaci\'on entre los nodos. Intentar agregar un nivel adicional de seguridad para la ubicaci\'on de la clave secreta.
 \item \textsl{Plan de Contingencia} : 
\end{itemize}

\textbf{Riesgo 5}
\begin{itemize}
 \item \textsl{Descripci\'on}: No se quiere que el volumen de la informaci\'on sea un problema. No sea cosa que se empiece a trabar todo cuando lleguen las fotitos de los drones o cuando crezcan los datos al crecer las plantitas.
 \item \textsl{Probabilidad}: Media
 \item \textsl{Impacto}: Alto	
 \item \textsl{Exposición}: Baja
 \item \textsl{Mitigación}: Utilizar una infraestructura de nodos distribuidos por todo el país. Utilizar alguna tecnica para guardar informacion geografica
 \item \textsl{Plan de Contingencia} : En el caso de volumen excesivo agregar mas nodos
\end{itemize}


\textbf{Riesgo 6}
\begin{itemize}
 \item \textsl{Descripci\'on}: Zonas relegadas por motivos politicos alegando motivos t\'ecnicos
 \item \textsl{Probabilidad}: Media
 \item \textsl{Impacto}: Alto
 \item \textsl{Exposición}: Baja
 \item \textsl{Mitigación}: Reportar el estado del sistema visualmente para evitar falsas acusaciones. Tambi\'en guardar registros de fallas t\'ecnicas
 \item \textsl{Plan de Contingencia} : En caso de que se presenten casos, se deberia auditar el sistema y utilizar los logs como prueba de que el sistema funciono correctamente. 
\end{itemize}


\textbf{Riesgo 7}
\begin{itemize}
 \item \textsl{Descripci\'on}: Pueden surgir errores en la comunicaci\'on con los servidores del \textbf{INTA}
 \item \textsl{Probabilidad}: Media
 \item \textsl{Impacto}: Media
 \item \textsl{Exposición}: Media
 \item \textsl{Mitigación}: Detectar fallas en la comunicación.
 \item \textsl{Plan de Contingencia} : Reportar las fallas al \textbf{INTA} para que a futuro se solucionen
\end{itemize}

\textbf{Riego 8}
\begin{itemize}
 \item \textsl{Descripci\'on}: En la zona patagónica no se puede garantizar una conectividad permanente de la red de datos entre los distintos artefactos afectados al sistema,
 \item \textsl{Probabilidad}: Alta
 \item \textsl{Impacto}: Media
 \item \textsl{Exposición}: Baja
 \item \textsl{Mitigación}: Permitir la carga manual de datos por medio de importación de archivos u otras fuentes.
 \item \textsl{Plan de Contingencia}: Considerar el uso de comunicaci\'on satelital para zonas de dif\'icil acceso. Para los dispositivos alguna t\'ecnologia alternativa de largo alcance.
\end{itemize}


\textbf{Riesgo 9}
\begin{itemize}
 \item \textsl{Descripci\'on}: 
 \item \textsl{Probabilidad}: 
 \item \textsl{Impacto}: 
 \item \textsl{Exposición}: 
 \item \textsl{Mitigación}: 
 \item \textsl{Plan de Contingencia}: 
\end{itemize}
