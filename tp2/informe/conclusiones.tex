\section{Conclusiones}
Desde el punto de vista del desarrollo tenemos pocas herramientas para comparar los métodos usados a lo largo de la materia (Scrum y UP),
dado que trabajando con Scrum realizamos una única iteración y trabajando con UP ni siquiera llegamos a desarrollar nada efectivamente.
Además, como cada método lo aplicamos a proyectos de tamaños distintos, la comparación que hagamos estará sesgada por cosas propias de la
complejidad de los problemas que quizás no forman parte de la metodología en sí, entorpeciendo nuestra capacidad de diferenciarlas únicamente
por sus características.

Sin embargo, sí pudimos comparar las etapas iniciales de ambas metodologías, viendo claramente una diferencia entre realizar una planificación
y un diseño antes de comenzar a programar; y empezar a programar sin planificación previa e ir diseñando a medida que se desarrolla. Si bien
en el primer TP consideramos que nuestra aplicación de Scrum era imperfecta pues estábamos acostumbrados a diseñar por adelantado,
en este segundo TP descubrimos que se puede planificar a un nivel más profundo del que conocíamos.

Desde un punto de vista subjetivo, podemos resaltar de UP la priorización de los atributos de calidad como ``drivers'' de las decisiones arquitectónicas,
entendiendo que con Scrum se podrían perder de vista en pos de resolver cuestiones funcionales, y que modificar una aplicación ya construida para que
satisfaga cierto atributo de calidad no contemplado previamente es mucho más costoso que diseñarla teniéndolo en cuenta desde un principio.

Sin embargo, preferimos Scrum a la hora de organizar los tiempos y responsabilidades del proyecto. Como ya dijimos no tuvimos la oportunidad de
experimentarlo nosotros mismos, pero consideramos que, por ejemplo, un error en la estimación de esfuerzo en Scrum tiene consecuencias mínimas
en comparación con las que tendría el mismo error (de estimación de horas/hombre) en UP: en Scrum como mucho una historia quedará afuera de una iteración y
tendrá que ser subdividida más adelante, mientras que en UP puede producir un efecto dominó que afecta todas las iteraciones posteriores
(porque ya se planificaron previamente) y eventualmente postergar todo el proyecto en caso de que la tarea sea crítica.

Entendemos, de todos modos, que combinar las características que destacamos de cada una de las metodologías es no-trivial. Afortunadamente, proponer una metodología
superadora excede el scope de este trabajo práctico.

A la hora de la comparación entre ``programming in the small'' y ``programming in the large'' podemos mencionar la diferencia entre las vistas más usadas: a la hora del 
diseño OO utilizamos tanto diagramas de clases (estático) como de objetos (dinámico) mientras que en la arquitectura se le da más importancia a la vista de
componentes y conectores (dinámica). Creemos que esto tiene que ver con que la arquitectura a gran escala se puede apreciar de manera más simple cuando vemos
cómo estarán los componentes ``en funcionamiento''.

Otra característica distintiva, aunque quizás sea un sesgo de parte de las metodologías usadas, es que la programación a pequeña escala está más orientada
a cumplir con atributos funcionales, mientras en que la programación a gran escala hay que darle mucha más relevancia a cuestiones no funcionales. Esto podría explicarse
si pensamos que en geneneral los productos más grandes son los que tienen más riesgos asociados relacionados con escalabilidad, seguridad, performance, etc.

Desde el punto de vista específico de la realización del trabajo, encontramos dificultades para diferenciar el análisis de riesgos de los atributos de calidad.
En mayor o menor medida, descubrimos que la mayoría de los atributos de calidad tienen algún riesgo relacionado (y viceversa), por lo que por momentos nos 
parecía redundante realizar ambos análisis. Quizás la diferencia se nota mejor en un proyecto real con stakeholders reales en el cual el análisis de riesgos
precede temporalmente a la documentación de atributos de calidad, pero de todos modos encontramos muchos puntos comunes a ambos análisis.