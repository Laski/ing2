

\begin{itemize}
 \item Monitoreo
 \begin{itemize}
    \item Obteniendo datos del \textbf{INTA}
    \begin{itemize}
      \item Cada un cierto intervalo el sistema consulta al \textbf{INTA} para obtener nuevos datos de  la red de micro-estaciones climatol\'ogicas.
    \end{itemize}
    \item Recibiendo images de los drones
    \begin{itemize}
	\item Los drones envian fotos cada cierto tiempo de las distintas areas y big cherry las recibe
    \end{itemize}
    \item Monitoreando estado de salud plantas
    \begin{itemize}
	\item Se le permite al usuario ver el estado de sus plantas segun calculos del plan e im\'agenes
    \end{itemize}
    \item Consultando mapa de campo
    \begin{itemize}
	\item El usuario puede ver el mapa calculado luego de procesar las imagenes de los drones.
    \end{itemize}
 \end{itemize}
 \item Interacci\'on Actuadores
 \begin{itemize}
  \item Enviando Accion a Actuador
  \begin{itemize}
   \item Para un determinado actuador de envia los comandos correspondientes para su uso.
  \end{itemize}

 \end{itemize}
 \item Seguimiento cultivos
 \begin{itemize}
  \item Consultando estado de cultivo
    \begin{itemize}
      \item Big cherry realiza calculos con toda la informaci\'on  y muestra en pantalla el estado actual del cultivo
    \end{itemize}
  \item Iniciando seguimiento en cultivo
  \begin{itemize}
	\item El usuario carga el nuevo cultivo para realizar el seguimiento
    \end{itemize}
  \end{itemize}
  \item Ingresando indicador de estado de cultivo
  \begin{itemize}
   \item Un Ing. Agr\'onomo ingresa datos de muestreo realizadas manualmente
  \end{itemize}

 \item Gesti\'on Planes
 \begin{itemize}
    \item Personalizando Plan
    \begin{itemize}
	\item El usuario puede usar algun plan existente y modificarlos a gusto.
    \end{itemize}
%     \item 
 \end{itemize}
 \item Supervisaci\'on
 \begin{itemize}
  \item Supervisando ordenes
  \begin{itemize}
    \item el usuario tiene una pantalla donde, por cada operaci\'on que esta por realizarse, se le permite cancelar. La opci\'on de cancelar expira con algun timeout 
  \end{itemize}
  \item Cancelando orden futura para actuador
  \begin{itemize}
    \item el usuario cancela una operacion.
  \end{itemize}
 \end{itemize} 
 \item Adminsitraci\'on General
 \begin{itemize}
    \item ABM Region 
    \begin{itemize}
	\item ABM de las regiones de cultivo o mejor conocidos como campos
    \end{itemize}
    \item ABM Cultivo
    \begin{itemize}
	\item ABM para los cultivos que se realizaran en las distintas regiones
    \end{itemize}
    \item ABM Actuadores
    \item ABM Drones
    \item ABM Sensores
    \item ABM Clientes
    \begin{itemize}
	\item Los usuarios consultores tienen acceso a este ABM para permitir manejar sus clientes.
    \end{itemize}
 \end{itemize}
 \item Adminsitraci\'on Super Usuario
 \begin{itemize}
    \item ABM Planes
    \begin{itemize}
	\item
    \end{itemize}
    \item ABM Usuario
    \item ABM Perfiles de Usuario
    \begin{itemize}
	\item Dado que el sistema debe soportar multiples perfiles el super usuario tiene control de estos por medio de este ABM.
    \end{itemize}
 \end{itemize}
 
 \item Auditor\'ia
 \begin{itemize}
    \item Revisando registro de decisiones tomadas
    \begin{itemize}
	\item Se permite acceder a un log donde se guardan todas las operaciones realizadas por big cherry
    \end{itemize}
    \item Consultando Gastos Red Privada
    \begin{itemize}
     \item El sistema muestra la contabilizaci\'on de las hect\'areas fotografiadas.
    \end{itemize}
    \item Consultando reporte de fallas con el \textbf{INTA}
    \begin{itemize}
     \item El sistema detecta fallas, se le permite al usuario ver un reporte de estas con este caso de uso.
    \end{itemize}

 \end{itemize}
 \item Autenticaci\'on
 \begin{itemize}
    \item Logueandose al sistema
    \begin{itemize}
      \item Dado que se require seguridad es necesario tener un control se autenticaci\'on. Este caso de uso representa la forma de autenticarse en big cherry
    \end{itemize}
    \item Deslogueandose del sistema
    \begin{itemize}
      \item Caso para cuando el usuario o client quiere salir del sistema.
    \end{itemize}
 \end{itemize}
\end{itemize}
