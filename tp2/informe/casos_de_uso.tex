\section{Introducción}

En el presente trabajo se aborda la extensión del diseño presentado en el TP1, debiendo ahora plantear un sistema de cultivo de muchas plantas en grandes extensiones de terreno y para muchos clientes. Dada esta nueva situación, aparecen varios stakeholders que, por su colaboración en el proyecto proveyendo logística o servicios, o por su intención de consumo, imponen distintos requerimientos que sumados a los técnicos desembocan en la contemplación de diversos riesgos y atributos de calidad.\\
\indent Aquí se presenta una planificación de las fases de \textit{elaboración} y \textit{construcción} de la metodología UP, asumiendo que la fase \textit{inception} ya se ha desarrollado. En esta primera etapa se han presentado las ideas para la formulación del proyecto BigCherry, se han establecido los requerimientos y recursos disponibles, y se ha llevado a cabo el QAW del cual se desprendieron algunos atributos de calidad esperados en el producto final. Como parte de la primera iteración, se identificaron los casos de uso más relevantes y en base a la recopilación de toda esta información se realizó un análisis de riesgos que se utilizó en parte para organizaron los CU en las distintas iteraciones incrementales. Además se diseñó la arquitectura del sistema, realizando una comparación con la del TP1, lo cual motivó algunas comparaciones sobre el contraste entre \textit{programming in the small vs. programming in the large}. 

\section{Casos de Uso}

Para comenzar a trabajar sobre la planificación del proyecto se identificó un conjunto de casos de uso que cubriera la mayoría de las funcionalidades que se desprenden de los requerimientos obtenidos. En un principio se los agrupó, tal y como se muestra a continuación, por su funcionalidad. Posteriormente, y en base al análisis de riesgos y a la información destilada de la fase de incepción, se utilizó este refinamiento para la organización y planificación de las iteraciones de elaboración y construcción.

\begin{itemize} \itemsep -2pt
    \item \textbf{Comunicación}
    \vspace{-7pt}
    \begin{itemize} \itemsep -2pt
        \item Accediendo a la red pública / privada de drones
		\item Interactuando con actuadores
		\item Organizando información almacenada entre nodos ArSat
        \item Recibiendo imágenes de los drones
		\item Descargando fotos offline de dron
    \end{itemize}
    \item \textbf{Monitoreo}
    \vspace{-7pt}
    \begin{itemize} \itemsep -2pt
        \item Monitoreando el estado de salud de las plantas
		\item Obteniendo información del clima a partir de las micro-estaciones climatológicas del INTA
		\item Visualizando el mapa del estado del terreno
		\item Consultando planes maestros de cultivo provistos por el INTA / org. privados (ABM Plan)
		\item Consultando estado de los cultivos
    \end{itemize}
    \item \textbf{Procesamiento}
    \vspace{-7pt}
    \begin{itemize} \itemsep -2pt
    	\item Procesando datos de bandas espectrales de las fotografías
		\item Generando mediciones de estado del terreno / cultivos a partir de fotos procesadas
        \item Calculando decisión según información recolectada y plan maestro
    \end{itemize}
    \item \textbf{Gestión}
    \vspace{-7pt}
    \begin{itemize} \itemsep -2pt
		\item Iniciando seguimiento de cultivos / región
		\item Incorporando nuevas especies de cultivo (ABM Cultivo)
		\item Agregando muestreo manual del estado de los cultivos / terreno
		\item Personalizando plan maestro de cultivo
		\item Agregando nuevos actuadores (ABM Actuador)
    \end{itemize}
    \item \textbf{Supervisión}
    \vspace{-7pt}
    \begin{itemize} \itemsep -2pt
    	\item Supervisando acciones ordenadas a actuadores
        \item Manejando fallas de servidores de INTA
    \end{itemize}
    \item \textbf{Seguridad}
    \vspace{-7pt}
    \begin{itemize} \itemsep -2pt
		\item Autenticando usuario (ABM Usuario)
		\item Autorizando usuario (ABMs Rol y Permiso)
		\item Encriptando / desencriptando información a almacenar
    \end{itemize}
    \item \textbf{Auditoría}
    \vspace{-7pt}
    \begin{itemize} \itemsep -2pt
    	\item Logueando eventos del sistema
		\item Revisando registro de eventos
    \end{itemize}
\end{itemize}

