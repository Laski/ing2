\section{Comparación con el TP1}

De la comparación de las figuras se observa el incremento de complejidad natural al pasar de un entorno controlado y con una sola planta 
a uno con varias plantas distintas, de diferentes tipos y ubicadas en diferentes lugares del país.

En primer lugar observamos que al tener directamente sensores en la tierra no necesitamos drones para tomar fotos ni componentes para analizar imágenes,
la información sobre el suelo es accesible directamente. 

De un modo similar se manifiesta la comunicación con la interfaz meteorológica, donde evitamos tener que precisar el lugar de origen de los datos, y los
componentes específicos para analizar los datos. 