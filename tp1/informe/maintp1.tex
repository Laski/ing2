\documentclass[a4paper,10pt]{article}

\usepackage[margin=1in]{geometry} 	% Setea el margen manualmente, todos iguales.
\usepackage[spanish]{babel} 		% {Con estos dos anda
\usepackage[utf8]{inputenc} 		% todo lo que es tildes y ñ}
\usepackage{fancyhdr} 			%{Estos dos son para
\usepackage{ulem}
\pagestyle{fancyplain} 			% el header copado}
\usepackage{color}			% Con esto puedo hacer la matufia de poner en color blanco un texto para engañar al formato
\usepackage{graphicx}	% Para insertar gráficos
\usepackage{array}			% Para usar arrays
\usepackage{hyperref}		% Para que tenga links el índice

%\usepackage{datetime}	% Para agregar automáticamente fecha/hora de compilación y otras cosas

\lhead{Teoría de Lenguajes} 	% {Con esto se usa el header copado. También está \chead para
\rhead{Graficador de Curvas 'Mylanga'} 	% el centro y comandos para el pie de página, buscar fancyhdr}
\renewcommand{\footrulewidth}{0.4pt}
\lfoot{Facultad de Ciencias Exactas y Naturales}
\rfoot{Universidad de Buenos Aires}
%\rfoot{\textit{}}
\usepackage{amsfonts}	% para simbolos de reales, naturales, etc. se usa \mathbb{•} y la letra
\usepackage{amsmath}	% para \implies
%\usepackage{algorithm}
%\usepackage{algorithmic}
\usepackage{caratula}
%%%%%%%%%%%%%%%%%%%%%%%%%%%%%%%%%%%%%
%      COMANDOS ÚTILES USADOS       %
%%%%%%%%%%%%%%%%%%%%%%%%%%%%%%%%%%%%%

% \section{title} 		Te hace un título ``importante'' en negrita, numerado. También está \subsection{title} y \subsubsection{title}.
% \begin{itemize}		Te hace viñetas.
%	\item esto es un item	Cambiar itemize por enumerate te hace una numeración.
% \end{itemize}

% \textbf{text} 		Te hace el texto en negrita (bold).
% \underline{text}		Te subraya el texto.

% \textsuperscript{text}	Te hace ``superindices'' con texto. En teoría subscript debería funcionar, pero se puede usar guion bajo entre llaves
% 				y signos peso para hacerlo como alternativa. Sino buscar.

% \begin{tabular}{cols} 	Es para hacer tablas. Se pone una c por cada columna deseada dentro de cols (si es que se desea centrada, l para justificar a 
%	a & b & c		izquierda, r a la derecha). Si se separa por espacios la tabla no tendrá líneas divisorias. Si se separa por | en lugar de 
% \end{tabular}			espacios, aparecerá una línea. Con || dos, y así. Luego para los elementos de las filas se escriben y se separan con ampersand (&).
%				Finalmente, para las líneas horizontales, se usa \hline para una linea en toda la tabla y \cline{i - j} te hace la linea desde
%				la celda i hasta la j, arrancando en 1.
%				Si en la columna se pone p(width) podés escribir un párrafo en la celda. Para hacer un enter con \\ no funciona porque te hace un
%				enter en la fila. Para eso se usa el comando \newline.
  
% \textcolor{color predefinido en palabras}{text}

%%%%%%%%%%%%%%%%%%%%%%%%%%%%%%%%%%%%%
%    FIN COMANDOS ÚTILES USADOS     %
%%%%%%%%%%%%%%%%%%%%%%%%%%%%%%%%%%%%%

\newcommand{\Gather}[1]{\begin{gather*}#1\end{gather*}}
%\newcommand{\Def}[1]{\textbf{Definición: }#1}
%\newcommand{\Prop}[1]{\textbf{Propiedad: }#1}
%\newcommand{\Teo}[1]{\textbf{Teorema: }#1}
\newcommand{\Obs}[1]{\textbf{Observación: }#1}
%\newcommand{\Amat}{A \in \mathbb{R}^{n\textnormal{x}n}}
\newcommand{\filtro}[1]{\textbf{\textit{#1}}}

\begin{document}

%%%%%%%%%%%%%%%%%%%%%%%%%%%
%			INICIO DE CARÁTULA			%
%%%%%%%%%%%%%%%%%%%%%%%%%%%

%% **************************************************************************
%
%  Package 'caratula', version 0.2 (para componer caratulas de TPs del DC).
%
%  En caso de dudas, problemas o sugerencias sobre este package escribir a
%  Nico Rosner (nrosner arroba dc.uba.ar).
%
% **************************************************************************



% ----- Informacion sobre el package para el sistema -----------------------

\NeedsTeXFormat{LaTeX2e}
\ProvidesPackage{caratula}[2003/4/13 v0.1 Para componer caratulas de TPs del DC]


% ----- Imprimir un mensajito al procesar un .tex que use este package -----

\typeout{Cargando package 'caratula' v0.2 (21/4/2003)}


% ----- Algunas variables --------------------------------------------------

\let\Materia\relax
\let\Submateria\relax
\let\Titulo\relax
\let\Subtitulo\relax
\let\Grupo\relax


% ----- Comandos para que el usuario defina las variables ------------------

\def\materia#1{\def\Materia{#1}}
\def\submateria#1{\def\Submateria{#1}}
\def\titulo#1{\def\Titulo{#1}}
\def\subtitulo#1{\def\Subtitulo{#1}}
\def\grupo#1{\def\Grupo{#1}}


% ----- Token list para los integrantes ------------------------------------

\newtoks\intlist\intlist={}


% ----- Comando para que el usuario agregue integrantes

\def\integrante#1#2#3{\intlist=\expandafter{\the\intlist
	\rule{0pt}{1.2em}#1&#2&\tt #3\\[0.2em]}}


% ----- Macro para generar la tabla de integrantes -------------------------

\def\tablaints{%
	\begin{tabular}{|l@{\hspace{4ex}}c@{\hspace{4ex}}l|}
		\hline
		\rule{0pt}{1.2em}Integrante & LU & Correo electr\'onico\\[0.2em]
		\hline
		\the\intlist
		\hline
	\end{tabular}}


% ----- Codigo para manejo de errores --------------------------------------

\def\se{\let\ifsetuperror\iftrue}
\def\ifsetuperror{%
	\let\ifsetuperror\iffalse
	\ifx\Materia\relax\se\errhelp={Te olvidaste de proveer una \materia{}.}\fi
	\ifx\Titulo\relax\se\errhelp={Te olvidaste de proveer un \titulo{}.}\fi
	\edef\mlist{\the\intlist}\ifx\mlist\empty\se%
	\errhelp={Tenes que proveer al menos un \integrante{nombre}{lu}{email}.}\fi
	\expandafter\ifsetuperror}


% ----- Reemplazamos el comando \maketitle de LaTeX con el nuestro ---------

\def\maketitle{%
	\ifsetuperror\errmessage{Faltan datos de la caratula! Ingresar 'h' para mas informacion.}\fi
	\thispagestyle{empty}
	\begin{center}
	\vspace*{\stretch{2}}
	{\LARGE\textbf{\Materia}}\\[1em]
	\ifx\Submateria\relax\else{\Large \Submateria}\\[0.5em]\fi
	\par\vspace{\stretch{1}}
	{\large Departamento de Computaci\'on}\\[0.5em]
	{\large Facultad de Ciencias Exactas y Naturales}\\[0.5em]
	{\large Universidad de Buenos Aires}
	\par\vspace{\stretch{3}}
	{\Large \textbf{\Titulo}}\\[0.8em]
	{\Large \Subtitulo}
	\par\vspace{\stretch{3}}
	\ifx\Grupo\relax\else\textbf{\Grupo}\par\bigskip\fi
	\tablaints
	\end{center}
	\vspace*{\stretch{3}}
	\newpage}





\materia{Ingeniería de Software 2}
\submateria{Segundo Cuatrimestre de 2014}
\titulo{Trabajo Práctico 1 \\ Entrega 1}

\grupo{Grupo}
\integrante{Giordano, Mauro}{125/10}{mauro.foxh@gmail.com}
\integrante{Iglesias, Axel}{79/10}{axeligl@gmail.com}
\integrante{Lascano, Nahuel}{476/11}{laski.nahuel@gmail.com}
\integrante{Lazzaro, Leonardo}{147/05}{lazzaroleonardo@gmail.com}

\begin{titlepage}
\maketitle
\thispagestyle{empty}
\end{titlepage} 

%%%%%%%%%%%%%%%%%%%%%%%%%%%
%				FIN DE CARÁTULA			%
%%%%%%%%%%%%%%%%%%%%%%%%%%%

%\tableofcontents
%\clearpage

%%%%%%%%%%%%%%%%%%%%%%%%%%%
%					DESARROLLO				%
%%%%%%%%%%%%%%%%%%%%%%%%%%%

\section{Contenido}

En esta preentrega se incluye el output del Sprint y Backlog de $stories$ restantes según la planificación inicial de esta etapa en la herramienta RallyDev. Además se incluye un detalle de los roles identificados y la estimación de los tiempos y costos asignados a la ejecución de dicho Sprint.

\section{Roles}

Durante el planificación de las $stories$ para el backlog y posterior $sprint$, definimos los siguientes roles para enfocar los distintos propósitos en la aplicación final.

\begin{itemize}
\item \textbf{Botánico}: Es quien se encarga de diseñar el plan de cultivo, crecimiento y cuidado de las plantas, pudiendo realizar cambios sobre la marcha si fuera necesario.
\item \textbf{Jardinero}: Es quien se encarga de llevar a la práctica el plan diseñado por el botánico, realizar las tareas manuales de jardinería y llevar un monitoreo más técnico que el del consumidor.
\item \textbf{Consumidor}: es el que paga la semilla, los insumos y es dueño de los cultivos. Su interés es que todo salga bien para poder hacer usufructo de los mismos, por lo tanto debe monitorear los gastos, el estado de las plantas y saber cuándo podrá cosechar.
\end{itemize}

\section{Estimación de tiempo/costo del Sprint}

El equipo de trabajo se conforma por los 4 integrantes, quienes le dedicaríamos 2 horas por día de cursada a la realización del primer tramo del proyecto. Como en el sprint serán 8 días hábiles, en total cada uno aportaría 16 horas. Estimamos que un 15\% de ese tiempo reservado se pierde fuera del propósito de desarrollo principal, lo que nos da 13,60 horas reales por cada integrante durante esta etapa. En total se trata de 54,4 horas humanas entre todo el equpo.\\
\indent Por otro lado, las stories seleccionadas para el sprint suman 40 story points según la estimación de esfuerzo realizada. Como no contamos con un historial de otros sprints, definimos arbitrariamente que por cada story point se deberán consumir 1,2 horas de trabajo. Esto da 48 horas en total, dejando así 6,4 horas de margen en el caso de que la estimación no se cumpliera tal cual (probablemente) o hubiera que agregar otra story al sprint.


\end{document}
