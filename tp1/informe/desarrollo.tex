\section{Contenido}

En esta preentrega se incluye el output del Sprint y Backlog de $stories$ restantes según la planificación inicial de esta etapa en la herramienta RallyDev. Además se incluye un detalle de los roles identificados y la estimación de los tiempos y costos asignados a la ejecución de dicho Sprint.

\section{Roles}

Durante el planificación de las $stories$ para el backlog y posterior $sprint$, definimos los siguientes roles para enfocar los distintos propósitos en la aplicación final.

\begin{itemize}
\item \textbf{Botánico}: Es quien se encarga de diseñar el plan de cultivo, crecimiento y cuidado de las plantas, pudiendo realizar cambios sobre la marcha si fuera necesario.
\item \textbf{Jardinero}: Es quien se encarga de llevar a la práctica el plan diseñado por el botánico, realizar las tareas manuales de jardinería y llevar un monitoreo más técnico que el del consumidor.
\item \textbf{Consumidor}: es el que paga la semilla, los insumos y es dueño de los cultivos. Su interés es que todo salga bien para poder hacer usufructo de los mismos, por lo tanto debe monitorear los gastos, el estado de las plantas y saber cuándo podrá cosechar.
\end{itemize}

\section{Estimación de tiempo/costo del Sprint}

El equipo de trabajo se conforma por los 4 integrantes, quienes le dedicaríamos 2 horas por día de cursada a la realización del primer tramo del proyecto. Como en el sprint serán 8 días hábiles, en total cada uno aportaría 16 horas. Estimamos que un 15\% de ese tiempo reservado se pierde fuera del propósito de desarrollo principal, lo que nos da 13,60 horas reales por cada integrante durante esta etapa. En total se trata de 54,4 horas humanas entre todo el equpo.\\
\indent Por otro lado, las stories seleccionadas para el sprint suman 40 story points según la estimación de esfuerzo realizada. Como no contamos con un historial de otros sprints, definimos arbitrariamente que por cada story point se deberán consumir 1,2 horas de trabajo. Esto da 48 horas en total, dejando así 6,4 horas de margen en el caso de que la estimación no se cumpliera tal cual (probablemente) o hubiera que agregar otra story al sprint.
